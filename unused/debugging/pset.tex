\documentclass{article}
\usepackage[utf8]{inputenc}

\title{\large{\textsc{Sorting Day 1: Quicksort and Heaps}}}
\date{}

\usepackage{natbib}
\usepackage{graphicx}
\usepackage{amsmath}
\usepackage{amsfonts}
\usepackage{mathtools}
\usepackage[a4paper, portrait, margin=0.8in]{geometry}

\usepackage{listings}


\newcommand\perm[2][n]{\prescript{#1\mkern-2.5mu}{}P_{#2}}
\newcommand\comb[2][n]{\prescript{#1\mkern-0.5mu}{}C_{#2}}
\newcommand*{\field}[1]{\mathbb{#1}}

\DeclarePairedDelimiter\ceil{\lceil}{\rceil}
\DeclarePairedDelimiter\floor{\lfloor}{\rfloor}

\newcommand{\Mod}[1]{\ (\text{mod}\ #1)}

\begin{document}

\maketitle

\subsection*{}

\begin{enumerate}
		
	%%%%% PROBLEM 1 %%%%%
	\item Earlier this semester we implemented a Priority Queue which could store integers. This PQ supported two methods: \texttt{enqueue} and \texttt{dequeueMax}. We implemented this data structure using an ArrayList, resulting in O(1) time for one operation, and O(N) for the other. Now, we would like to create a PriorityQueue implemented with a Heap. Additionally, we want this PQ to support enqueing and dequeueing of key-value pairs (not just int keys).
	      	      
	      \begin{itemize}
	      	\item \texttt{enqueue(K key, V value)} enqueues a key-value pair to the priority queue
	      	\item \texttt{dequeueMax()} removes, and returns, the key-value pair with the highest key on the queue.
	      \end{itemize}
	      	      
	      Find an instructor and explain how you would implement these methods (no code necessary) using a heap. Explain what the runtime of both methods is. (They should both be better than O(N)).
	      	      
	      %%%%% PROBLEM 2 %%%%%
	\item Given an unsorted length-N array \texttt{A}, find the \textbf{k}th largest element in the array. Assume $N > k > logN$. Can you do better than O(NlogN)?
	      	      
	      %%%%% PROBLEM 3 https://leetcode.com/problems/find-k-pairs-with-smallest-sums/?tab=Description %%%%%
	\item Given two integer arrays \texttt{A} and \texttt{B} sorted in ascending order, and a value \textbf{k}, a pair \textbf{(u,v)} is defined as consisting one element from \texttt{A} and one from \texttt{B}. Find \textbf{k} pairs with the smallest sum.
	      	      
	      Given \texttt{A} = $[1,7,11]$, \texttt{B} = $[2,4,6]$, \textbf{k} = 3, return $[1,2], [1,4], [1,6]$.
	      	      
	      Given \texttt{A} = $[1,1,2]$, \texttt{B} = $[1,2,3]$, \textbf{k} = 2, return $[1,1], [1,1]$
	      	      
	      Given \texttt{A} = $[1,2]$, \texttt{B} = $[3]$, \textbf{k} = 3, return all possible pairs $[1,3], [2,3]$
	      	      
	\item You have a mixed pile of N nuts and N bolts and need to quickly find the corresponding pairs of nuts and bolts. Each nut matches exactly one bolt, and each bolt matches exactly one nut. By fitting a nut and bolt together, you can see which is bigger. But it is not possible to directly compare two nuts or two bolts to see which is bigger. Your goal is to sort the array of nuts from smallest-to-largest, and sort the array of bolts from smallest-to-largest, so that you can determine which nut matches which bolt.
	      	      
\end{enumerate}

\clearpage

\begin{enumerate}
		
	\item For the priority queue, use a heap as the underlying data structure. Store Pair(key, value) objects, and keep the heap ordered by key. To enqueue an item, place it at the bottom of the heap. Then, follow the folling \texttt{float} procedure:
	      	      
	      \begin{enumerate}
	      	\item Compare the element with its parent; if they are in the correct order, we are done. If not, swap the element with it's parent
	      	\item Return to the previous step, but consider the parent as the current element.
	      \end{enumerate}
	\item Solution
	      \begin{lstlisting}[language=Java]
public int findKthLargest(int[] nums, int k) {

    final PriorityQueue<Integer> pq = new PriorityQueue<>();
    for(int val : nums) { // O(N)
        pq.offer(val); O(logk)

        if(pq.size() > k) {
            pq.poll(); O(logk)
        }
    }
    return pq.peek();
}

	      \end{lstlisting}

	\item Solution

	Assume you have the following class declaration:

	      \begin{lstlisting}[language=Java]
class Pair {
    int indexA;
    int indexB;
    Pair(indexA, indexB) {
        // set attributes
    }
 }
	      \end{lstlisting}
	      \begin{lstlisting}[language=python]
# Runtime is O(klogk)
SMALLEST_K_PAIRS(A, B, k):
    answers = []
    pq = minPQ()
    # use sum as key. add pairs as values to PQ
    # use top row of possible sums as input to PQ. Make sure pq never grows
    # larger than size k. Once it is larger than k, dequeue the min and place
    # it in answers.
    for i in 0:len(B):
        pq.enqueue(A[0] + B[i], new Pair(0, i))
        if pq.size() > k:
            p = pq.dequeueMin()
            answers.add(p)
            if p.indexB < B.length - 1:
		pq.enqueue(A[p.indexA] + B[p.indexB + 1], new Pair(p.indexA, p.indexB + 1))
        if answers.length == k:
            return answers
    while answers.length < k:
        p = pq.dequeueMin()
        answers.add(p)
        if p.indexB < B.length - 1:
	    pq.enqueue(A[p.indexA] + B[p.indexB + 1], new Pair(p.indexA, p.indexB + 1))

    return answers
}
	      \end{lstlisting}
	      
	\item Nuts and Bolts explanation (from https://www.geeksforgeeks.org/nuts-bolts-problem-lock-key-problem/):
	      
	      This algorithm first performs a partition by picking last element of bolts array as pivot, rearrange the array of nuts and returns the partition index ‘i’ such that all nuts smaller than nuts[i] are on the left side and all nuts greater than nuts[i] are on the right side. Next using the nuts[i] we can partition the array of bolts. Partitioning operations can easily be implemented in O(n). This operation also makes nuts and bolts array nicely partitioned. Now we apply this partitioning recursively on the left and right sub-array of nuts and bolts. As we apply partitioning on nuts and bolts both so the total time complexity will be O(2*nlogn) = O(nlogn) on average.
	      
\end{enumerate}

\end{document}
