
\documentclass{article}
\usepackage[utf8]{inputenc}

\title{\large{\textsc{In-Class 2: Linked Lists, Stacks and Queues}}}
\date{}

\usepackage{natbib}
\usepackage{graphicx}
\usepackage{amsmath}
\usepackage{amsfonts}
\usepackage{mathtools}
\usepackage[a4paper, portrait, margin=0.8in]{geometry}

\usepackage{listings}


\newcommand\perm[2][n]{\prescript{#1\mkern-2.5mu}{}P_{#2}}
\newcommand\comb[2][n]{\prescript{#1\mkern-0.5mu}{}C_{#2}}
\newcommand*{\field}[1]{\mathbb{#1}}

\DeclarePairedDelimiter\ceil{\lceil}{\rceil}
\DeclarePairedDelimiter\floor{\lfloor}{\rfloor}

\newcommand{\Mod}[1]{\ (\text{mod}\ #1)}

\begin{document}

\maketitle


\begin{lstlisting}[language=Java]
Node {
    int data;
    Node next;
}

\end{lstlisting}

\noindent For each of the following problems, assume you are only given a \texttt{Node head} (or two), which is the head of a linked list. Focus on first getting a working solution, and then optimizing to a O(N) time O(1) space.

\begin{enumerate}

\setcounter{enumi}{1}

\item Find the length of a linked list.

\item Find and return the $n^{th}$ node from the end of a linked list (inputting 1 would return the last node. Inputting 2 would return the node before the last node. etc.).

\begin{figure}[h!]
\includegraphics[width=4cm]{Y-ShapedLinked-List}
\centering
\end{figure}

\item Given two singly linked lists, determine if and where they have an intersection point. Return the intersection point node. Return \texttt{null} if there is no intersection. For example, for the above linked list, the function would return the Node whose value is 15.

\end{enumerate}

\noindent Finally, some doubly-linked list problems!

\begin{enumerate}

\begin{lstlisting}[language=Java]
Node {
    int data;
    Node next, prev;
}
\end{lstlisting}

\setcounter{enumi}{8}

\item Given a DLL, print the elements in reverse order

\item Check if a DLL is a palindrome.

    
\end{enumerate}

\subsection*{}

Assumptions:

\begin{itemize}
  \item All stacks are listed from top to bottom.
  \item All problems can be solved using O(N) time and O(N) space unless noted otherwise.
  \item Stacks support: \texttt{push}, \texttt{pop}, \texttt{peek}, and \texttt{isEmpty}. They have no other methods.
\end{itemize}
    
\subsection*{}

\begin{enumerate}

\item Given a stack, return a stack with all the elements in reverse order

\item Given two integer sequences without duplicates, if the first is the push sequence of a stack, determine if the second is a possible pop sequence of the same stack. Ex: if \texttt{[1, 2, 3, 4, 5]} is a push sequence, \texttt{[4, 5, 3, 2, 1]} is a possible pop sequence (Push 1, 2, 3, 4. Pop 4. Push 5. Pop 5, 3, 2, 1), but \texttt{[4, 3, 5, 1, 2]} is not. Drawing an example stack illustration could be helpful.

\item Given a string of open and close parenthesis, determine the longest substring that contains a valid expression. eg
\begin{itemize}
  \item \texttt{"(()))("} would return 4.
  \item \texttt{"(()()))()"} would return 6.
  \item \texttt{"))(("} would return 0. 
\end{itemize}
If you want an additional challenge, restrict yourself to using only one stack as additional storage.

\item Sort a stack in ascending order (with smallest items on top). The only extra memory you may use is one additional stack to hold items (you cannot use any arrays). Return the sorted stack. O(N\textsuperscript{2}) time.

\item Write a function to solve the Tower of Hanoi puzzle. Your function header should look like: \texttt{solveHanoi(Stack left, Stack middle, Stack right, int n)}, Discs are represented as integers, where smaller ints must be stacked on top of larger ints. \texttt{middle}, and \texttt{right} are empty to start, while \texttt{left} has \texttt{[1, 2, ..., n]} on the stack. Move all the ints over to \texttt{right} using only the four supported \texttt{Stack} operations. Larger ints may never be placed on top of smaller ints. Find an instructor and discuss the runtime complexity of your solution.
\end{enumerate}


\clearpage

\begin{lstlisting}[language=Python]
def swap(arr, i, j):
    temp = arr[i]
    arr[i] = arr[j]
    arr[j] = temp

# Find either the min or max element's index, starting from start
def index_find(arr, func, start=0):
    m = start
    for i in range(start+1, len(arr)):
        if func=='min' and arr[i] < arr[m]:
            m = i
        if func=='max' and arr[i] > arr[m]:
            m = i
    return m

# O(N^2) time
def min_pairs(arr):
    f_map = {0: 'min', 1: 'max'}
    # alternate placing minimum and maximum elements
    for i in range(len(arr)):
        swap(arr, i, index_find(arr, f_map[i%2], i))
    return arr


# If you're curious, here's the NlogN time solution
def min_pairs_faster(arr):
    arr = sorted(arr) # O(NlogN)
    for i in range(1, len(arr)/2, 2): # O(N)
        swap(arr, i, len(arr)-i)
    return arr

def equal(head1, head2):

    while (head1 is not None) and (head2 is not None):
        if head1.data != head2.data:
            return False
        head1 = head1.next
        head2 = head2.next

    return (head1 is None and head2 is None)

def reverseList(head):
    prev = None
    while head:
        curr = head
        head = head.next
        curr.next = prev
        prev = curr
    return prev

def hasCycle(head):
    slow = head
    fast = head
    while fast!=None and fast.next!=None:
        slow = slow.next
        fast = fast.next.next
        if slow is fast:
            return True
    return False

def find_nth(head, n):
    while n > 0 and head is not None:
        head = head.next
        n -= 1
    return head
    
def find_nth_from_end(head, n):
    fast = head
    slow = head
    for i in range(n):
        fast = fast.next
    while fast is not None:
        fast = fast.next
        slow = slow.next
    return slow

def length(head):
    l = 0
    while head is not None:
        head = head.next
        l+=1
    return l

def palindrome(head):

    l = length(head)

    if l <= 1:
        return True

    left_of_middle = find_nth(head, l/2 - 1)
    # if linked list has odd number of elements, drop the middle element
    if (l % 2) == 1:
        left_of_middle.next = left_of_middle.next.next

    # reverse the back half of the linked list
    left_of_middle.next = reverseList(left_of_middle.next)

    # seperate the front and back half of the LL into seperate Lists
    middle = left_of_middle.next
    left_of_middle.next = None

    # check equality
    return equal(head, middle)

def intersection(head1, head2):
    l1 = length(head1)
    l2 = length(head2)
    long_list, short_list = head1, head2
    if l2 > l1:
        long_list, short_list = head2, head1
    long_list = find_nth(long_list, abs(l1-l2))
    while long_list is not None:
        if long_list == short_list:
            return long_list
        long_list = long_list.next
        short_list = short_list.next
    return None

def print_rev(head):
    last_elem = find_nth(length(head) - 1)
    while last_elem is not None:
        print last_elem
        last_elem = last_elem.prev

def palindrome_dll(head):
    l = length(head)
    if l <= 1:
        return True
    last_elem = find_nth(l - 1)
    while head is not None:
        if head != last_elem:
            return False
        head = head.next
        last_elem = last_elem.prev
    return True

#Start of Stacks and Queues

def merge_ll(head1, head2):
    # Base cases
    if head1 is None:
        return head2
    if head2 is None:
        return head1

    # Keep a pointer to the head of the new list
    new_ll = head1

    # Use the correct head as the first element
    if head2.data < head1.data:
        new_ll = head2
        head2 = head2.next
    else:
        head1 = head1.next

    # Iterate until both heads are None, selecting the smaller
    # element and adding onto the end of the list
    new_head = new_ll
    while not (head1 is None and head2 is None):
        if (head1 is None) or ((head2 is not None) and (head1.data > head2.data)):
            new_head.next = head2
            head2 = head2.next
        else:
            new_head.next = head1
            head1 = head1.next
        new_head = new_head.next
    return new_ll

def num_palindrome(i):
    j = i
    stack = Stack()
    while j != 0:
        stack.push(j%10)
        j/=10
    while not stack.isEmpty():
        if stack.pop() != i%10:
            return False
        i/=10
    return True

def reverse(stack):
    rev = Stack()
    while not stack.isEmpty():
        rev.push(stack.pop())
    return rev

def valid_expr(s):
    stack = Stack()
    for c in s:
        if not stack.isEmpty() and c == ')' and stack.peek() == '(':
            stack.pop()
        else:
            stack.push(c)
    return stack.isEmpty()

def is_pop_sequence(s1, s2):
    s3 = Stack()
    while not s1.isEmpty():
        s3.push(s1.pop())
        while not s3.isEmpty() and s3.peek() == s2.peek():
            s3.pop()
            s2.pop()
    return s3.isEmpty()

def sorted_stack_one_stack(stack):
    sorted_stack = Stack()
    while not stack.isEmpty():
        var = stack.pop()
        while not sorted_stack.isEmpty() and sorted_stack.peek() < var:
            stack.push(sorted_stack.pop())
        sorted_stack.push(var)
    return sorted_stack

def longest_valid_substring(s):
    stack = Stack()
    for i, c in enumerate(s):
        if not stack.isEmpty() and c == ')' and s[stack.peek()] == '(':
            stack.pop()
        else:
            stack.push(i)
    end_valid = len(s)
    start_valid = len(s)
    longest = 0

    for i in range(len(s) - 1, -1, -1):
        if not stack.isEmpty() and stack.peek() == i:
            end_valid = i
            stack.pop()
        else:
            start_valid = i
            if end_valid - start_valid > longest:
                longest = end_valid - start_valid

    return longest

def solveHanoiHelper(fr, to, other, n):
    if n == 1:
        to.push(fr.pop())
    else:
        solveHanoiHelper(fr, other, to, n-1)
        solveHanoiHelper(fr, to, other, 1)
        solveHanoiHelper(other, to, fr, n-1)
        
def solveHanoi(left, middle, right, n):
    solveHanoiHelper(left, right, middle, n)

\end{lstlisting}

\end{document}
