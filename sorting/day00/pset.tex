\documentclass{article}
\usepackage[utf8]{inputenc}

\title{\large{\textsc{In-Class 5: Sorting}}}
\date{}

\usepackage{natbib}
\usepackage{graphicx}
\usepackage{amsmath}
\usepackage{amsfonts}
\usepackage{mathtools}
\usepackage[a4paper, portrait, margin=0.8in]{geometry}

\usepackage{listings}


\newcommand\perm[2][n]{\prescript{#1\mkern-2.5mu}{}P_{#2}}
\newcommand\comb[2][n]{\prescript{#1\mkern-0.5mu}{}C_{#2}}
\newcommand*{\field}[1]{\mathbb{#1}}

\DeclarePairedDelimiter\ceil{\lceil}{\rceil}
\DeclarePairedDelimiter\floor{\lfloor}{\rfloor}

\newcommand{\Mod}[1]{\ (\text{mod}\ #1)}

\begin{document}

\maketitle

\subsection*{}

For each of the following problems, write a function. Start with the CORE problems, but if you finish before the end of the class period you can attempt the EXTENSION and REVIEW problems.

\subsection*{}

CORE:

\begin{enumerate}

\item Suppose you have an array A[] of N distinct items which is nearly sorted: each item at most k positions away from its position in the sorted order. Design an algorithm to sort the array in O(Nlogk) time.
	
\item Given a list of \texttt{Interval}s as defined in the \texttt{Interval} class below, merge overlapping \texttt{Interval}s and output the resulting list of mutually exclusive \texttt{Interval}s.  For example:

\begin{itemize}
    \item Given the list of \texttt{Interval}s \texttt{[(2,4), (5,7), (1,3), (6,8)]}, the  intervals \texttt{(1,3)} and \texttt{(2,4)} overlap with each other, so they should be merged and become \texttt{(1,4)}. \texttt{(5,7)} and \texttt{(6,8)} also overlap, creating \texttt{(5,8)}.  The final solution for this list of \texttt{Interval}s would be \texttt{[(1,4), (5,8)]}.
\end{itemize}

The definition for the Intervals class is as follows:

\begin{lstlisting}[language=Java]
 public class Interval {
    int start;
    int end;
    Interval(int s, int e) { start = s; end = e; }
 }
\end{lstlisting}
    
\item Given the head of a linked list, sort it in place using O($n$log$n$) time and O(logN) space.
    
\end{enumerate}

\subsection*{}

EXTENSION:

\begin{enumerate}

\item Given two sorted arrays, merge them in a sorted manner (so that the time complexity is not O(n1 * n2), if n1 and n2 are the lengths of the input arrays).

\item Given an an unsorted array, sort the given array. You are allowed to do only following operation on array.

\texttt{flip(arr, i): Reverse array from 0 to i}

Let's say we are in a parallel universe where the flip operation is O(1) (but all other operations are the same time complexity). Write an efficient program for sorting a given array in O(nLogn) time on the given machine. 


\end{enumerate}

\subsection*{}

REVIEW:

\begin{enumerate}

\item Given a the head of a singly linked list, write a function to reverse every k nodes (where k is an input to the function). For example:
\begin{itemize}
\item Given 1 $\xrightarrow{}$ 2 $\xrightarrow{}$ 3 $\xrightarrow{}$ 4 $\xrightarrow{}$ 5 $\xrightarrow{}$ 6 $\xrightarrow{}$ 7 $\xrightarrow{}$ 8 $\xrightarrow{}$ NULL and k = 3, return 3 $\xrightarrow{}$ 2 $\xrightarrow{}$ 1 $\xrightarrow{}$ 6 $\xrightarrow{}$ 5 $\xrightarrow{}$ 4 $\xrightarrow{}$ 8 $\xrightarrow{}$ 7 $\xrightarrow{}$ NULL
\item Given 1 $\xrightarrow{}$ 2 $\xrightarrow{}$ 3 $\xrightarrow{}$ 4 $\xrightarrow{}$ 5 $\xrightarrow{}$ 6 $\xrightarrow{}$ 7 $\xrightarrow{}$ 8 $\xrightarrow{}$ NULL and k = 5, return 5 $\xrightarrow{}$ 4 $\xrightarrow{}$ 3 $\xrightarrow{}$ 2 $\xrightarrow{}$ 1 $\xrightarrow{}$ 8 $\xrightarrow{}$ 7 $\xrightarrow{}$ 6 $\xrightarrow{}$ NULL.
\end{itemize}
    
\item Given an array of N integers, return the number of positions for which A[i+1] $>$ A[i] after an optimal rearrangement. For example:
\begin{itemize} 
\item Given {20, 30, 10, 50, 40}, return 4. The optimal rearrangement is {50, 40, 30, 20, 10}.
\item Given {200, 100, 100, 200}, return 2. The optimal rearrangement is {100, 200, 100, 200}.
\end{itemize}
    
\item Given an array, return true if the array has a duplicate element in it.

\end{enumerate}

\clearpage

\begin{lstlisting}[language=Python]

# A is sorted from [lo, mid) and [mid, hi). Sorts elements from [lo, hi)
def merge(A, lo, mid, hi):
    i = lo
    j = mid
    B = []
    while (i < mid) or (j < hi):
        if (i == mid) or (j < hi and (A[j] < A[i])):
            B.append(A[j])
            j += 1
        else:
            B.append(A[i])
            i += 1
    for k in range(lo, hi):
        A[k] = B[k - lo]

# Sort element in [lo, hi)
def mergeSort(A, lo, hi):
    if hi - lo < 2:
        return
    mid = (hi + lo) / 2
    mergeSort(A, lo, mid)
    mergeSort(A, mid, hi)
    merge(A, lo, mid, hi)


def sortK(A, k):
    # first, sort from [0, 2k). After this, first k smallest elements are in
    # the correct position. then sort from [k, 3k). Now, the first 2k smallest
    # elements are in the correct position. Continue across the entire array.
    # This inner loop runs (N/k) times. Each loop iteration performs a
    # mergesort on a length-k array, which is complexity klogk. Therefore,
    # overall complexity is Nlogk.
    for i in range(0, len(A), k):
        mergeSort(A, i, min(i + 2*k, len(A)))
        
        
def mergeArrays(arr1, arr2, n1, n2): 
    arr3 = [None] * (n1 + n2) 
    i = 0
    j = 0
    k = 0
    while i < n1 and j < n2: 
        if arr1[i] < arr2[j]: 
            arr3[k] = arr1[i] 
            k = k + 1
            i = i + 1
        else: 
            arr3[k] = arr2[j] 
            k = k + 1
            j = j + 1
    while i < n1: 
        arr3[k] = arr1[i]; 
        k = k + 1
        i = i + 1
    while j < n2: 
        arr3[k] = arr2[j]; 
        k = k + 1
        j = j + 1
        
        
def insertionSort(arr): 
    for i in range(1,len(arr)): 
        j = insertionSortHelper(arr, 0, i-1, arr[i]) 
        if (j != -1): 
            flip(arr, j-1) 
            flip(arr, i-1) 
            flip(arr, i) 
            flip(arr, j) 
            
def insertionSortHelper(arr,low,high,x): 
    if x <= arr[low]: 
        return low 
    if x > arr[high]: 
        return -1
    mid = (low + high)/2  #low + (high-low)/2 
    if(arr[mid] == x): 
        return mid 
    if(arr[mid] < x): 
        if(mid + 1 <= high and x <= arr[mid+1]): 
            return mid + 1
        else: 
            return insertionSortHelper(arr, mid+1, high, x) 
    if (mid - 1 >= low and x > arr[mid-1]): 
        return mid 
    else: 
        return insertionSortHelper(arr, low, mid - 1, x) 

\end{lstlisting}

\begin{lstlisting}[language=Java]
MERGE_INTERVALS(intervals) {
    intervals.sortByStartTime();
    
    // use an iterator here
    for (interval i : intervals) {
        while (i.end > i.next.start) {
            // merge sets i.end to max(i.end, i.next.end)
            merge(i, i.next);
        }
    }
}

SORT_LINKED_LIST(linkedlist) {
    leftList, rightList = split(linkedlist)
    SORT_LINKED_LIST(leftList)
    SORT_LINKED_LIST(rightList)
    merge(leftList, rightList)
    }
}

Node reverse(Node head, int k) { 
   Node current = head; 
   Node next = null; 
   Node prev = null; 
   int count = 0; 
   while (count < k && current != null)  { 
       next = current.next; 
       current.next = prev; 
       prev = current; 
       current = next; 
       count++; 
   } 
   if (next != null)  
      head.next = reverse(next, k); 
   return prev; 
}

static int countMaxPos(int[] arr) { 
    int n = arr.length; 
    HashMap<Integer, Integer> map 
        = new HashMap<Integer, Integer>(); 
    for (int x : arr) { 
        if (map.containsKey(x)) 
            map.put(x, map.get(x) + 1); 
        else
            map.put(x, 1); 
    } 
    int max_freq = 0; 
    for (Map.Entry entry : map.entrySet()) 
        max_freq = Math.max(max_freq, (int)entry.getValue()); 
    
    return n - max_freq; 
} 

\end{lstlisting}

\begin{lstlisting}[language=Python]

def contains_duplicate(A):
    # O(N) time, O(N) space
    s = set()
    for i in A:
        if i in s:
            return False
        s.add(i)

\end{lstlisting}


\end{document}
